\documentclass[11pt]{article}
\setlength{\oddsidemargin}{0in}
\setlength{\evensidemargin}{0in}
\setlength{\textwidth}{6.5in}
\setlength{\parindent}{0in}
\setlength{\parskip}{\baselineskip}

\usepackage{amsmath,amsfonts,amssymb}
\usepackage{fancybox}
\usepackage{subfiles}
\usepackage{enumitem, tabularx, booktabs, ragged2e}
\usepackage[margin=1in]{geometry}
\usepackage{listings, lstautogobble}
\usepackage{alltt}
\usepackage{tikz}
\usepackage{tabularx}
\usepackage{graphicx}
\usepackage{xcolor}
\usepackage{blkarray}
\usepackage[bottom]{footmisc}
\usepackage{rotating}
\usepackage{wasysym}
\usepackage{rotating}
\usepackage[hidelinks]{hyperref}
\graphicspath{{images/}{./img/}}
\usepackage{verbatimbox}
\usepackage{tablefootnote}
\newcolumntype{L}{>{\centering\arraybackslash}m{1cm}}

% colors in math environment
\usepackage{xcolor}
\definecolor{orange}{rgb}{1,0.5,0}
\definecolor{blue}{rgb}{0.22, 0.58, 0.82}
\definecolor{green}{rgb}{0.2, 0.65, 0.32}
\definecolor{red}{rgb}{0.91, 0.26, 0.2}
\definecolor{purple}{rgb}{0.46, 0.21, 0.68}
\makeatletter
\def\mathcolor#1#{\@mathcolor{#1}}
\def\@mathcolor#1#2#3{%
	\protect\leavevmode
	\begingroup
	\color#1{#2}#3%
	\endgroup
}
\makeatother

\allowdisplaybreaks

% Label subfigures as 1(a) instead of 1a
\usepackage[labelformat=simple]{subcaption}
\renewcommand\thesubfigure{(\alph{subfigure})}
\usetikzlibrary{automata,positioning}

% Var, MSE, Bias
\newcommand{\Var}{\text{Var}}
\newcommand{\Cov}{\text{Cov}}
\newcommand{\MSE}{\text{MSE}}
\newcommand{\bias}{\text{Bias}}

% Norm and absolute value
\newcommand{\norm}[1]{\left\lVert#1\right\rVert}
\newcommand{\abs}[1]{\left\lvert#1\right\rvert}

% argmax and argmin
\DeclareMathOperator*{\argmin}{argmin} % no space, limits underneath in displays
\DeclareMathOperator*{\argmax}{argmax} % no space, limits underneath in displays

% Decorators
\newcommand{\dinkus}{\begin{center}***\end{center}}

% Make code look nicer
\definecolor{codegreen}{rgb}{0,0.6,0}
\definecolor{codegray}{rgb}{0.5,0.5,0.5}
\definecolor{codepurple}{rgb}{0.58,0,0.82}
\definecolor{backcolour}{rgb}{0.95,0.95,0.92}
\lstdefinestyle{mystyle}{
    backgroundcolor=\color{backcolour},   
    commentstyle=\color{codegreen},
    keywordstyle=\color{magenta},
    numberstyle=\tiny\color{codegray},
    stringstyle=\color{codepurple},
    basicstyle=\ttfamily\footnotesize,
    breakatwhitespace=false,         
    breaklines=true,                 
    captionpos=b,                    
    keepspaces=true,                 
    numbers=left,                    
    numbersep=5pt,                  
    showspaces=false,                
    showstringspaces=false,
    showtabs=false,                  
    tabsize=2
}
\lstset{style=mystyle}

\begin{document}

\begin{center}
  \setlength\fboxsep{0.5cm}
  \fbox{\parbox{\textwidth}{
  \textbf{STAT xxx: Title} \hfill \textbf{Spring xxxx}
 \begin{center}
	 {\Large\textbf{Homework 01}} \\
	 Due xxx xx, xxxx by 11:59 PM \\
 \end{center}
	\textit{Instructor: xxx xxx} \hfill \textit{TA: Apara Venkat}
	}}
\end{center}



\begin{comment}

	% Using subfiles
	\subfile{questions/q1}
	
	
	% Import a code file
	\lstinputlisting[language=R]{code/q1.R}
	
	
	% Appendix
	\appendix
	\section{Code}
	
	
	% Single figure
	\begin{figure}[!tbh]
		\centering
		\includegraphics[height=3in]{image.png}
		\caption{Image}
		\label{fig:img}
	\end{figure}

	
	% Side-by-side figure each with captions and labels
	\begin{figure*}[!tbh]
    	\centering
		
		% Figure (a)
    	\begin{subfigure}[t]{0.5\textwidth}
        	\centering
	        \includegraphics[height=2in]{image_a.png}
    	    \caption{Image (a)}
        	\label{fig:img-a}
	    \end{subfigure}%
    	~ 
	
		% Figure (b)
	    \begin{subfigure}[t]{0.5\textwidth}
    	    \centering
        	\includegraphics[height=2in]{image_b.png.png}
	        \caption{Image (b)}
    	    \label{fig:img-b}
	    \end{subfigure}
	    
	    % Caption for combined figure
	    \caption{Image}
    	\label{fig:img}
	\end{figure*}
	
	
	% Simple table
	% Use https://tablesgenerator.com/
	% Python: https://pandas.pydata.org/docs/reference/api/pandas.DataFrame.to_latex.html
	% R: http://xtable.r-forge.r-project.org/
	\begin{table}[!tbh]
		\centering
		\begin{tabular}{cccc}
			\hline
			\textbf{Column 1} & \textbf{Column 2} & \textbf{Column 3} & \textbf{Column 4} \\
			\hline
			\hline
			1 & & & \\
			2 & & & \\
			\hline
		\end{tabular}
		\caption{Table.}
		\label{tab:table}
	\end{table}
	
\end{comment}



\end{document}

